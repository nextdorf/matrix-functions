%\documentclass{article}
\documentclass[12pt]{article}

\usepackage[a4paper, total={185mm, 265mm}]{geometry} %A4: 210px * 297px

\usepackage[utf8]{inputenc} % Required for inputting international characters
\usepackage{babel}
\usepackage[T1]{fontenc} % Output font encoding for international characters

\usepackage{amssymb}
\usepackage{amsmath}
\usepackage{bbm}
%\usepackage{mathpazo} % Use the Palatino font by default
%\usepackage{pazocal}
\usepackage{mathtools}
\usepackage{physics}

\usepackage{xcolor}
\usepackage{subcaption}
\usepackage{float}
\usepackage{shellesc}
\usepackage{svg}
\usepackage[linkcolor=blue, colorlinks=true]{hyperref}

\usepackage[backend=bibtex,style=authoryear,natbib=true]{biblatex} % Use the bibtex backend with the authoryear citation style (which resembles APA)

\addbibresource{literature.bib} % The filename of the bibliography

\usepackage[autostyle=true]{csquotes} % Required to generate language-dependent quotes in the bibliography
\usepackage{color}


\newcommand{\argmin}{\operatornamewithlimits{argmin}}
\newcommand{\argmax}{\operatornamewithlimits{argmax}}
\renewcommand{\i}{{i\mkern1mu}}
\newcommand{\del}{\partial}
\newcommand{\cev}[1]{\reflectbox{\ensuremath{\vec{\reflectbox{\ensuremath{#1}}}}}}

\newcommand{\pushright}[1]{\ifmeasuring@#1\else\omit\hfill$\displaystyle#1$\fi\ignorespaces}
\newcommand{\pushleft}[1]{\ifmeasuring@#1\else\omit$\displaystyle#1$\hfill\fi\ignorespaces}

\begin{document}



\centerline{\sc \large Documentation for Matrixfuncs}
\vspace{.5pc}
\vspace{2pc}

\tableofcontents

\section{Mathematical background}
\subsection{2 dimensional case}
The Cayley-Hamilton theorem states for $A\in \mathbb C^{n\times n}:\ p_A(A)=0$\\
For $n=2$:
\begin{align}
p_A(\lambda) =& \det(\lambda\mathbbm 1 - A) = (\lambda - \lambda_1)(\lambda - \lambda_2) = \lambda^2 -\lambda\tr A + \det A\\
\Rightarrow A^2 =& A \tr A - \mathbbm 1 \det A
\end{align}
So $A^2$ is a linear combination of $A$ and $\mathbbm 1$. By complete induction it follows that $A^n$ is also such a linear combination of $A$ and $\mathbbm 1$ for any $n \in \mathbbm N_0$:
\begin{align}
A^n \eqqcolon a_n A + b_n\mathbbm 1,\ \ \ A^{n+1} = \big(a_n\tr A + b_n\big)A - \big(a_n\det A \big)\mathbbm 1
\end{align}
\begin{align}
a_{n+1} = a_n\tr A + b_n,\ b_{n+1} = -a_n\det A,\ a_0=0,\ a_1=1,\ b_0=1,\ b_1=0 \label{recurrent 2dim}
\end{align}
In order to solve the recurrence equation we use a shifting operator $\Delta$ for $a_n$, s.t. $a_{n+1} = \Delta a_n$
\begin{align}
0 = \big(\Delta^2 - \Delta\tr A + \det A\big) a_n = p_A(\Delta) a_n = (\Delta-\lambda_1)(\Delta-\lambda_2) a_n
\end{align}
As ansatz we choose $a_n$ as a linear combination of the solutions of $0=(\Delta-\lambda_1)a_n$ and $0=(\Delta-\lambda_2)a_n$. In general this ansatz only works if $\lambda_1 \neq \lambda_2$. We will consider the case $\lambda_1 = \lambda_2$ again in the end of this section and in section $\ref{ch:math.general case}$:
\begin{align}
a_n =& c_1 \lambda_1^n + c_2\lambda_2^n
\end{align}
As induction start we can simply consider the trivial cases $n=0$ and $n=1$:
\begin{align}
a_0 =& c_1 + c_2 = 0,\ a_1 = c_1\lambda_1 + c_2\lambda_2 = 1 \nonumber\\
\Rightarrow c_2 =& -c_1,\ c_1(\lambda_1 - \lambda_2) = 1 \nonumber\\
\Rightarrow c_1 =& -c_2 = \frac 1 {\lambda_1 - \lambda_2}
\end{align}
We found the solutions for $a_n$:
\begin{align}
a_n = \frac {\lambda_1^n - \lambda_2^n}{\lambda_1 - \lambda_2},\ b_n \overset{\eqref{recurrent 2dim}}=-\frac {\lambda_2\lambda_1^n - \lambda_1\lambda_2^n}{\lambda_1 - \lambda_2}
\end{align}
and thereby the linear combination of $A^n$:
\begin{align}
A^n =& \frac {\lambda_1^n - \lambda_2^n}{\lambda_1 - \lambda_2} A - \frac {\lambda_2\lambda_1^n - \lambda_1\lambda_2^n}{\lambda_1 - \lambda_2} \mathbbm 1
\end{align}
Since $A^n$ is linear in $\lambda_i^n$ we know how to evaluate arbitrary polynomials of $A$ in the krylov space. Let $q(x)$ be a polynomial:
\begin{align}
q(A) =& \frac 1{\lambda_1 - \lambda_2} \left[(q(\lambda_1) - q(\lambda_2))A - (\lambda_2 q(\lambda_1) - \lambda_1 q(\lambda_2)) \mathbbm 1\right]
\end{align}
Since the coefficients only depend on $p(\lambda_i)$ the formula can be generalized to all functions which are analytical in $\lambda_1$ and $\lambda_2$. Let $f(x)$ be a analytical in the eigenvalues of $A$:
\begin{align}
f(A) =& \frac 1{\lambda_1 - \lambda_2} \left[(f(\lambda_1) - f(\lambda_2))A - (\lambda_2 f(\lambda_1) - \lambda_1 f(\lambda_2)) \mathbbm 1\right] \label{func 2dim}
\end{align}
The generalization of $\eqref{func 2dim}$ is the core of this package. The important properties are that it yields the result to arbitrary precision in constant time and that the application of $f$ commutes with any linear function applied to the krylov space. With other words this means there exists a rank-3 tensor $F$, which only depends on $A$, s.t. $f(A)_{ij} = F_{ijk} f(\lambda_k)$.\\
Since $f(A)$ is continuous in the eigenvalues and the set of square matrices with unequal eigenvalues is dense in the set of all square matrices, we can just consider the limit of $\lambda_1 \to \lambda_2$ to calculate the case $\lambda_1 = \lambda_2$:
\begin{align}
f(A|\lambda_1&=\lambda+\epsilon,\; \lambda_2 = \lambda) = \frac 1\epsilon \left[ \epsilon f'(\lambda) A - \epsilon(\lambda f'(\lambda) - f(\lambda))\mathbbm 1 \right] + o(\epsilon) \nonumber\\
\overset{\epsilon\to 0}{\longrightarrow}& f'(\lambda)\big( A-\lambda \mathbbm 1 \big) + f(\lambda)\mathbbm 1\nonumber\\
\Rightarrow f(A) =& f'(\lambda)\big( A-\lambda \mathbbm 1 \big) + f(\lambda)\mathbbm 1\ \ \text{if $\lambda$ is the only eigenvalue of $A$}  \label{func 2dim 1eigval}
\end{align}
So when $A$ is not diagonalizable $f(A)$ depends on $f'(\lambda)$. In general we will find that $f(A)$ depends on $f(\lambda_i), f'(\lambda_i), \dots, f^{(m)}(\lambda_i)$ where $m$ is the difference of the algebraic and geometric multiplicity of $\lambda_i$.
\subsection{Example for 2 dimensions}
There are many applications and possible use cases for matrix functions, but the main application this package is written for are difference equations. Assuming you want to consider a sequence of vectors $(v_i)_{i=0}^\infty \subset \mathbbm K^n$ which satisfies $v_{i+1} = M v_i$ for some matrix $M\in \mathbbm K^{n\times n}$. An obvious conclusion is $v_k = M^k v_0$. With $\eqref{func 2dim}$ or $\eqref{func 2dim 1eigval}$ we can express $M^k$ without any matrix multiplications. Instead with just apply $f(\lambda) = \lambda^k$ to all eigenvalues. We even can apply $v_0$ or any other tensor contraction to $f(M)$ before we specify $k$. Calculating this things beforehand can speedup the total calculation if we need to calculate $v_k$ for many different $k$.\\
Another neat application is the analytical continuation of $v_k$ in $k$ by setting $f(\lambda) = e^{k \ln(\lambda)}$. We will use this in the example here by solving a difference equation for a sampled $\sin$ function and then evaluating the numerically solved $\sin$ between the sampled points.\\
Consider the sum identity of $\sin$:
\begin{align}
\sin(x+a) = \sin(x)\cos(a) + \cos(x)\sin(a) \label{sin(a+b)}
\end{align}
For fixed $a$ $\sin(x+a)$ can be expressed as a linear combination of the basis $\sin x$ and $\cos b$. Therefore $\sin x$ can be written as a linear combination of $\sin(x+a)$ and $\sin(x+2a)$ for almost all $a$:
\begin{align}
\sin(x) \eqqcolon& \alpha\sin(x+a) + \beta\sin(x+2a) \nonumber\\
=& \alpha\big(\sin(x)\cos(a) + \sin(a)\cos(x)\big) + \beta\big(\sin(x)\cos(2a) + \sin(2a)\cos(x)\big) \nonumber\\
=& \sin x\big( \alpha\cos a + \beta\cos(2a) \big) + \cos x\big( \alpha\sin a + \beta\sin(2a) \big) \\
\Rightarrow \begin{pmatrix} 1 \\ 0 \end{pmatrix} =& \begin{pmatrix} \cos a & \cos(2a) \\ \sin a & \sin(2a) \end{pmatrix} \begin{pmatrix} \alpha \\ \beta \end{pmatrix}\\
\begin{pmatrix} \alpha \\ \beta \end{pmatrix} =& \begin{pmatrix} \cos a & \cos(2a) \\ \sin a & \sin(2a) \end{pmatrix}^{-1} \begin{pmatrix} 1 \\ 0 \end{pmatrix} \nonumber\\
=& \frac 1{\cos a\sin(2a) - \sin a\cos(2a)} \begin{pmatrix} \sin(2a) & -\cos(2a) \\ -\sin a & \cos a \end{pmatrix} \begin{pmatrix} 1 \\ 0 \end{pmatrix} \nonumber\\
=& \frac 1{\sin a} \begin{pmatrix} \sin(2a) \\ -\sin a \end{pmatrix} \nonumber\\
\overset{\eqref{sin(a+b)}}=& \begin{pmatrix} 2\cos(a) \\ -1 \end{pmatrix} \nonumber
\end{align}
From this we can construct a difference equation which shifts $\sin$ by $a$. 
\begin{align}
\begin{pmatrix} \sin x \\ \sin(x - a) \end{pmatrix} =& \underbrace{\begin{pmatrix} 2\cos a & -1 \\ 1 & 0 \end{pmatrix}}_{\eqqcolon M(a)} \begin{pmatrix} \sin (x-a) \\ \sin(x-2a) \end{pmatrix} \nonumber\\
\Rightarrow \begin{pmatrix} \sin (x+na) \\ \sin(x+(n-1)a) \end{pmatrix} =& M(a)^n \begin{pmatrix} \sin x \\ \sin(x-a) \end{pmatrix}\\
\Rightarrow \sin(x+na) =& \hat e_1 \cdot M(a)^n \begin{pmatrix} \sin x \\ \sin(x-a) \end{pmatrix}\\
\Rightarrow \sin(x+y) =& \hat e_1 \cdot e^{\frac ya \ln(M(a))} \begin{pmatrix} \sin x \\ \sin(x-a) \end{pmatrix}
\end{align}
We can now use the formula for two dimensions $\eqref{func 2dim}$ in order to solve for $\sin(x+y)$:
\begin{align}
\sin(x+y) \overset{\eqref{func 2dim}}=& \hat e_1 \cdot \frac 1{\lambda_1 - \lambda_2} \left[(f(\lambda_1) - f(\lambda_2))M(a) - (\lambda_2 f(\lambda_1) - \lambda_1 f(\lambda_2)) \mathbbm 1\right] \begin{pmatrix} \sin x \\ \sin(x-a) \end{pmatrix} \label{sin(x+y) numerical}
\end{align}
\begin{align*}
\text{with}\ f(\lambda) =& e^{\frac ya \ln\lambda}\\
\lambda_{1,2} =& \frac 12\tr M(a) \pm \sqrt{\big(\frac 12\tr (M(a))\big)^2 - \det(M(a))}\nonumber\\
=& \cos a \pm \sqrt{\cos^2 a - 1}\nonumber\\
=& e^{\pm \i a}\\
f(\lambda_{1,2}) =& e^{\pm \i y}
\end{align*}
Simplifying $\eqref{sin(x+y) numerical}$ will yield $\eqref{sin(a+b)}$.
\subsection{General case}\label{ch:math.general case}
Cayley-Hamilton: $A \in \mathbb C^{n\times n}:\ p_A(A) = 0$
\begin{align}
p_A(\lambda) =& \det(\lambda\mathbbm 1 - A) = \prod_{k=1}^n(\lambda - \lambda_k) = \lambda^n - \sum_{k=0}^{n-1} \Lambda_k \lambda^k\nonumber\\
\Rightarrow \Lambda_k =& \sum_{\substack{j_1,\dots,j_{n-k} = 1,\\ j_1<\dots<j_{n-k}}}^n (-1)^{n-k+1} \prod_{i=1}^{n-k} \lambda_{j_i} \\
A^n =& \sum_{k=0}^{n-1} \Lambda_k A^k\nonumber\\
A^m \eqqcolon& \sum_{k=0}^{n-1} \alpha_{mk} A^k\\
A^{m+1} =& \sum_{k=0}^{n-2} \alpha_{mk} A^{k+1} + \alpha_{m,n-1} A^n \nonumber\\
=& \sum_{k=1}^{n-1} (\alpha_{m,k-1} + \alpha_{m,n-1}\Lambda_k) A^k + \alpha_{m,n-1}\Lambda_0 A^0 \nonumber\\
\end{align}
\begin{align}
\alpha_{m+1,0} = \alpha_{m,n-1} \Lambda_0,\ \alpha_{m+1,k} = \alpha_{m,k-1} + \alpha_{m,n-1}\Lambda_k \label{recurrent ndim}
\end{align}
\begin{align}
\alpha_{m,n-1} =& \sum_{k=1}^{n-1} \alpha_{m-k,n-1}\Lambda_{n-k} + \alpha_{m+1-n,0} \nonumber\\
=& \sum_{k=1}^n \alpha_{m-k,n-1}\Lambda_{n-k} \nonumber\\
=& \sum_{k=0}^{n-1} \alpha_{m+k-n,n-1}\Lambda_k\\
\Rightarrow 0=& \alpha_{m+n,n-1} - \sum_{k=0}^{n-1} \alpha_{m+k,n-1}\Lambda_k \nonumber\\
=& \big( \Delta^n - \sum_{k=0}^{n-1} \Delta^k \Lambda_k \big) \alpha_{m, n-1} \ \ \ \text{with}\ \Delta \alpha_{mk} = \alpha_{m+1,k}\nonumber\\
=& p_A(\Delta) \alpha_{m, n-1}
\end{align}
General ansatz, similiar to the two dimensional case:
\begin{align}
\alpha_{m,n-1} = \sum_{k=1}^n \beta_k \lambda_k^m
\end{align}
For $m<n$:
\begin{align}
\alpha_{mk} =& \delta_{mk}\\
\Rightarrow \delta_{m,n-1} =& \beta_k \lambda_k^m\\
\Rightarrow \vec \beta \perp& \sum_{k=1}^n \hat e_k \lambda_k^m \eqqcolon \vec \lambda^{(m)} \ \text{for}\ m=0\dots n-2\\
\tilde\beta_k \coloneqq& \det(\hat e_k | \vec\lambda^{(n-2)} | \vec\lambda^{(n-3)} | \dots | \vec\lambda^{(0)})\nonumber\\
\beta =& \tilde\beta /\big( \tilde \beta \cdot \lambda^{(n-1)} \big)
\end{align}
$\eqref{recurrent ndim}$ yields for $\alpha_{mk}$:
\begin{align}
\alpha_{mk} =& \sum_{j=1}^{k+1} \alpha_{m-j,n-1} \Lambda_{k+1-j} \nonumber\\
=& \sum_{j=1}^{k+1} \vec\beta\cdot\vec\lambda^{(m-j)} \Lambda_{k+1-j} \nonumber\\
=& \vec\beta\cdot \sum_{j=0}^k \Lambda_j \vec\lambda^{(m+j-k-1)}
\end{align}
As in the two dimensional case this defines $f(A)$ if $f$ is analytical in the eigenvalues of $A$:
\begin{align}
A^m =& \sum_{k=0}^{n-1} A^k \sum_{j=0}^k \Lambda_j  \beta \cdot \lambda^{(m+j-k-1)}\\
f(A) =& \sum_{k=0}^{n-1} A^k \sum_{j=0}^k \Lambda_j \sum_{i=1}^n \beta_i \lambda_i^{j-k-1} f(\lambda_i)
\end{align}

\subsubsection*{Sanity check}
\begin{align*}
\Lambda_0 =& -\det(A),\ \Lambda_1 = \tr A,\ \beta = \frac 1 {\lambda_1 - \lambda_2}\begin{pmatrix} 1 \\ -1 \end{pmatrix}\\
f(A) =& \mathbbm 1 \Lambda_0 \frac 1{\lambda_1 - \lambda_2} \big(f(\lambda_1)/\lambda_1 - f(\lambda_2)/\lambda_2 \big)\\
&+ A \Lambda_0 \frac 1{\lambda_1 - \lambda_2} \big(f(\lambda_1)/\lambda_1^2 - f(\lambda_2)/\lambda_2^2 \big)\\
&+ A \Lambda_1 \frac 1{\lambda_1 - \lambda_2} \big(f(\lambda_1)/\lambda_1 - f(\lambda_2)/\lambda_2 \big)\\
=& \frac 1{\lambda_1-\lambda_2}\big[ \lambda_1f(\lambda_2) - \lambda_2f(\lambda_1) \big] \mathbbm 1 \\
&+ \frac 1{\lambda_1-\lambda_2}\big[ (\frac{\lambda_1}{\lambda_2} - \frac{\lambda_1+\lambda_2}{\lambda_2}) f(\lambda_2) - (\frac{\lambda_2}{\lambda_1} - \frac{\lambda_1+\lambda_2}{\lambda_1}) f(\lambda_1) \big] A\\
=& \frac 1{\lambda_1-\lambda_2}\big[ \big( \lambda_1f(\lambda_2) - \lambda_2f(\lambda_1) \big) \mathbbm 1 + \big(f(\lambda_1) - f(\lambda_2)\big) A\big] 
\end{align*}
\include{matrix-functions-docu-appendix}


\end{document}

